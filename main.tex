% =============================================================================
% Compile parameters
% =============================================================================

\documentclass[12pt, russian]{extarticle}
\usepackage[T2A]{fontenc}
\usepackage{fontspec}
\defaultfontfeatures{Ligatures={TeX},Renderer=Basic}
\setmainfont[Ligatures={TeX,Historic}]{Times New Roman}
\usepackage[a4paper,
left=25mm,
right=15mm,
top=20mm,
bottom=20mm]{geometry}
\usepackage[linktoc=all]{hyperref}
\usepackage{titlesec}
\titlelabel{\thetitle.\quad}
\usepackage{tocloft}

% Remove \bfseries from section titles in ToC
\renewcommand{\cftsecfont}{}
% Remove \bfseries from section titles' page in ToC
\renewcommand{\cftsecpagefont}{}
\renewcommand{\cftsecaftersnum}{.}
\usepackage{titlesec}

% Change font size for types
\titleformat*{\section}{\large\bfseries}
\titleformat*{\subsection}{\large\bfseries}
\titleformat*{\subsubsection}{\large\bfseries}

\setlength{\parindent}{1.25cm}
\setlength{\parskip}{0.4cm}
\font\subtitlefont=cmr12 at 12pt
\font\titlefont=cmr12 at 24pt
\usepackage{color}
\usepackage{mathtools}
\usepackage{listings}
\usepackage{graphicx}
\usepackage{tocloft}
\usepackage{indentfirst}
\usepackage{enumitem}
\usepackage{graphicx}
\usepackage{subcaption}
\usepackage[russian]{babel}
\usepackage{setspace}
\setlength{\marginparwidth}{2cm}
\usepackage{todonotes}
\usepackage{tikz}
\usepackage{pdfpages}
\usetikzlibrary{shapes, arrows, positioning}
\renewcommand{\contentsname}{}
\renewcommand{\cftsecleader}{\cftdotfill{\cftdotsep}}
\graphicspath{ {./resources/} }

\usepackage{listings}
\usepackage{float}

\lstset{
    numbers=left,                   % where to put the line-numbers
    numberstyle=\small \ttfamily \color[rgb]{0.4,0.4,0.4},
                % style used for the linenumbers
    showspaces=false,               % show spaces adding special underscores
    showstringspaces=false,         % underline spaces within strings
    showtabs=false,                 % show tabs within strings adding particular underscores
    frame=lines,                    % add a frame around the code
    tabsize=4,                        % default tabsize: 4 spaces
    breaklines=true,                % automatic line breaking
    breakatwhitespace=false,        % automatic breaks should only happen at whitespace
    basicstyle=\ttfamily,
    %identifierstyle=\color[rgb]{0.3,0.133,0.133},   % colors in variables and function names, if desired.
    keywordstyle=\color[rgb]{0.133,0.133,0.6},
    commentstyle=\color[rgb]{0.133,0.545,0.133},
    stringstyle=\color[rgb]{0.627,0.126,0.941},
}

% =============================================================================
% Useful 
% =============================================================================

% insert code
% \lstinputlisting[language=Python]{./resources/classification.py}


% insert image
% \begin{figure}[H]
%     \centering
%     \includegraphics[scale=0.6]{resources/2.png}
% \end{figure}

% =============================================================================
% End of compile parameters
% =============================================================================

\title{}
\author{}
\date{}

\begin{document}

% =============================================================================
% Global titlepage
% =============================================================================

\begin{titlepage}

    \begin{center}
        МИНИСТЕРСТВО НАУКИ И ВЫСШЕГО ОБРАЗОВАНИЯ РОССИЙСКОЙ ФЕДЕРАЦИИ \\
        Федеральное государственное автономное образовательное учреждение \\
        высшего образования \\
        \textbf{
            «Национальный исследовательский \\
            Нижегородский государственный университет им. Н.И. Лобачевского»\\ (ННГУ)
        }
    \bigbreak

    \vspace{2em}
        \textbf{
            Институт информационных технологий, математики и механики
            \bigbreak
            Кафедра математического обеспечения и суперкомпьютерных технологий
        }

        Направление подготовки: «Программная инженерия» \\
        Профиль подготовки: «Технологии цифровой трансформации»

        \bigbreak
        \bigbreak
        \bigbreak

        \textbf{ОТЧЕТ} \\
        по ознакомительной практике
        \bigbreak

        на тему \\
        {\bfseries ``Решение задачи классификации с использованием методов машинного обучения''}
    \end{center}

    \vspace{5em}

    \begin{flushright}
        {\bfseries Выполнил:} студент группы \\ 3824М1ПР1 Д.Э. Булгаков\\
        \hfill Подпись \hspace{5em} \newline \\
        {\bfseries Проверил:} \\доцент кафедры МОСТ, к.т.н.,\\ Н.А. Борисов \\
        \hfill Подпись \hspace{5em} \newline \\
    \end{flushright}


    \vspace{\fill}

    \begin{center}
        Нижний Новгород\\2024
    \end{center}

\end{titlepage}

% =============================================================================
% Main content
% =============================================================================

% ========== Set global spacing ==========
\begin{spacing}{1.5}

% ========== Table of content ==========
\tableofcontents
\thispagestyle{empty}
\newpage

% Params to make the following text start with
% its page number
\pagestyle{plain}
\setcounter{page}{3}


\section{Введение}

\newpage
\section{Постановка задачи}


\newpage
\section{Проведенная работа}


\newpage
\section{Выводы}

\newpage
\section{Список литературы}

\end{spacing}
\end{document}
